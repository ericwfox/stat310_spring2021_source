\documentclass[11pt]{beamer}
\usepackage{amsmath}
\usepackage{amssymb}
\usepackage{geometry}
\usepackage{graphicx}
\usepackage{url}
\usepackage{xcolor}

% some latex magic for correcting apostrophe issue in verbatim mode
\makeatletter
\let \@sverbatim \@verbatim
\def \@verbatim {\@sverbatim \verbatimplus}
{\catcode`'=13 \gdef \verbatimplus{\catcode`'=13 \chardef '=13 }} 
\makeatother

\begin{document}

%---------------------------------------------
\begin{frame}
\large
Lecture 6:\\
Sampling Distributions\\
STAT 310, Spring 2021
\normalsize
\end{frame}

%---------------------------------------------
\begin{frame}
\begin{itemize}
\item A \textbf{parameter} is a numerical characteristic of the population (fixed number that is usually unknown).\\ 
\vspace{10pt}
\item A \textbf{statistic} is a numerical characteristic of the sample (varies depending on sample). \\
\vspace{10pt}
\item The statistic is also referred to as a \textbf{point estimate}, since it is our best guess at the value of a population parameter.  
\end{itemize}
\end{frame}

%---------------------------------------------
\begin{frame}
% picutures
\end{frame}

%---------------------------------------------
\begin{frame}
\large
Notation for the proportion:
\begin{itemize}
\item $p$: population proportion
\item $\hat{p}$: sample proportion\\
\end{itemize}
\vspace{15pt}

Notation for the mean:
\begin{itemize}
\item $\mu$: population mean
\item $\bar{x}$: sample mean\\
\end{itemize}
\vspace{15pt}

The sample size is denoted as $n$
\end{frame}

%---------------------------------------------
\begin{frame}{Example}
\small
A recent Gallup poll found that 66\% of Americans are dissatisfied with how the COVID-19 vaccination process is going in the U.S.  The survey results were based on random sample of 4,098 adults.\footnote{\tiny \url{https://news.gallup.com/poll/329552/two-thirds-americans-not-satisfied-vaccine-rollout.aspx}}
\begin{enumerate}[(a)]
\item What is the sample proportion?
\vspace{0.5cm}
\item What is the sample size?
\vspace{0.5cm}
\item Describe the population proportion?
\vspace{1.25cm}
\item Suppose another poll is conducted with a different random sample of adults.  Would you expect the sample proportion to be the same, or slightly different?
\vspace{0.5cm}
\end{enumerate}
\end{frame}

%---------------------------------------------
\begin{frame}
\begin{itemize}
\item \textbf{Sampling Error} refers to how much a statistic, such as the sample proportion, will vary from one random sample to the next.
\vspace{10pt}
\item For example, the Gallup poll reported a sampling error of $\pm 2$ percentage points.  This means that the population proportion of Americans that are dissatisfied with the vaccine rollout is likely between 64\% and 68\%. 
\end{itemize}
\end{frame}

%---------------------------------------------
\begin{frame}
\vspace{-3.5cm}
A \textbf{sampling distribution} is the distribution of a statistic when repeatedly taking random samples from a population.

\end{frame}

%---------------------------------------------
\begin{frame}
\end{frame}

%---------------------------------------------
\begin{frame}
\begin{itemize}
\item In real-world applications, we never actually observe the sampling distribution, since we usually take a single random sample.
\vspace{10pt}
\item However, it is useful to always think of a statistic, such as the sample proportion, as coming from such a hypothetical distribution.
\vspace{10pt}
\item The concept of a sampling distribution is very important when trying to quantify sampling error.
\end{itemize}
\end{frame}

%---------------------------------------------
\begin{frame}{Central Limit Theorem (CLT)}
The sampling distribution for $\hat{p}$ follows an approximate normal distribution centered around the population proportion $p$, and with standard error $\sqrt{p (1-p) / n}$.\\
\vspace{4.5cm}

%This is one of the results of the so-called Central Limit Theorem.  The mathematical proof is quite involved an requires advanced probability.  However, the theorem can be easily verified using computer simulations.\\  
\end{frame}

%---------------------------------------------
\begin{frame}{Conditions for CLT}
The following conditions should be met to apply the CLT:\\
\vspace{5pt}
\begin{itemize}
\item The data come from a simple random sample.  This is called the \textbf{independence condition} since it implies that the individuals or cases in the data are unrelated.
\vspace{5pt}
\item $np \geq 10$ and $n(1-p) \geq 10$.  This is sometimes called the \textbf{success-failure condition}  since $np$ can be interpreted as the expected number of successes and $n(1-p)$ the expected number of failures.
\end{itemize}
\end{frame}

%---------------------------------------------
\begin{frame}{Example}
\small
Suppose that the population proportion of Americans who support the expansion of solar energy is $p=0.88$, and $n=1000$ Americans are randomly sampled.
\begin{enumerate}[(a)]
\item What is the mean, or center, of the sampling distribution for $\hat{p}$?
\vspace{0.75cm}
\item What is the standard error of the sampling distribution for $\hat{p}$?
\vspace{1cm}
\item What distribution does $\hat{p}$ follow?
\vspace{1cm}
\item Are the conditions for the CLT satisfied?
\vspace{1.5cm}
\end{enumerate}
\end{frame}

%---------------------------------------------
\begin{frame}{Example}
\small
Suppose that population proportion of Americans who support the expansion of solar energy is $p=0.88$, and $n=1000$ Americans are randomly sampled.  What is the probability that the sample proportion $\hat{p}$ will be greater than 0.9?
\vspace{6cm}
\end{frame}




\end{document}